% intro.tex:

\chapter{Introduction}
\label{chap:intro}
This is where the intro goes. 
There is a planet ... where people still believe digital watches are the 
greatest invention. It is mostly harmless. 

\section{Life, Universe and Everything}
\label{chap:intro:design}

Simple, quantitative answers to the most pondering questions are beautiful. More about question of life, universe and everything can be found in \cite{book:42}.

This is where the text goes. One can refer to a previous chapter like \chapref{chap:intro}. To find all the other reference possibilities search in the MACROS folder for chapref. Also one can include pictures, preferably in the pdf format like shown in \figref{fig:intro:meth}.

This demonstrates how to include a term\index{Term} into the index\index{Index}. For details, check out \cite{latex:index}.

\begin{figure}[htbp]
  \centering
    \includegraphics[width=0.5\textwidth]{fig/meth.pdf}
  \caption[Design methodology for SoC design]{\label{fig:intro:meth} Design methodology for SoC design (Source \cite{book:SpecC:yellow})}
\end{figure}

Also tables are possible as shown in \tabref{tab:ConclusionSummary}. What is really cool are the acronyms. The first time you use one, like \ac{TLM} and \ac{AHB} then it appears with full description. Using it the second time makes only the acronym appears \ac{TLM} and \ac{AHB}.

\begin{table}[htbp]
  \centering
     \begin{tabular}{|l|c|}
        \hline 
         \textbf{Environment Condition} & \textbf{Applicable Model} \\
			\hline
			 \begin{tabular}{@{}l@{}}
			 $\bullet\,$ single master bus \\
			 $\bullet\,$ no overlap between masters bus access\\
			 \end{tabular}
			&
			 \ac{TLM}
			\\
			\hline
			 \begin{tabular}{@{}l@{}}
			 $\bullet\,$ only locked transfers \\
			 $\bullet\,$ unlocked transfers and low overlap\\
			 \end{tabular}
			&
			 \ac{ATLM}
			\\
			\hline
			 \begin{tabular}{@{}l@{}}
			 $\bullet\,$ unlocked transfers and high overlap\\
			 \end{tabular}
			&
			 bus functional
			\\
			\hline
		\end{tabular}
	\caption{Conclusion summary}
	\label{tab:ConclusionSummary}
\end{table}



% --- EOF ---
